\chapter{Context of the internship}
\label{ch:intro}

\section{Company}


\section{Product}
The company's main software, Whistle, is designed to handle up to a billion calls per month on about six virtualized (or not) servers and can connect seamlessly to run on or with Rackspace clouds, Amazon’s clouds or on a private cluster of machines. Instead of paying a penny or so per minute to a VoIP company, businesses that want to add voice calling over the web to their social network, their app or their role-playing game just deploy this software and take care of it themselves.

Whistle, the company's next-generation communications engine's core platform was built for scale. Your media servers, configuration databases and call handling logic can all be spread across the internet in any way you see fit. Fully utilize all that the Cloud has to offer. Its entire management and configuration capabilities are exposed via simple APIs. Did we mention, it's free? Welcome to the next generation of VoIP.

The plan is to offer folks the source code for Whistle, but to support and sell services on top of it, in much the same that every other open source software company plans to make money. Schreiber says the services include a voice mail platform, a minute reseller platform, a conference call serve and will include many more. 
 
\section{About the internship}
The intern joined a team of 2 senior software engineers, who each have 10+ years of experience in programming and telephony systems architecture. The role of the intern was to contribute to the development of Whistle, developing Erlang module and being fully integrated in the Whistle team.
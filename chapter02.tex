\chapter{Agile development}
\label{ch:startup}

\section{Introduction to Agile practices}
The vast majority of software projects suffer from a steady degradation of design quality and it becomes more and more difficult to maintain the software with the same level of quality. As the software ages it calcifies and becomes harder and harder to maintain. In some cases it becomes too expensive to maintain and so it the software is put to rest and rewritten. In others, the software is released with a steadily increasing number of defects. Both of these common situations are deeply unsatisfying, but there are several practices that can be set up to reduce defects, improve design and increase quality.

Many of the practices from the Agile world stop the degradation of software quality and turn the trend around. It is not unheard of for teams to have maintained a zero-defect status for months and years. Design and architecture have become malleable; they now emerge and transform over time. 

Below are some practices used by the company the student worked for.

\section{Refactoring}
The practice of Refactoring code changes the structure (i.e., the design) of the code while maintaining its behavior. Incremental improvement of design is the name of the game with refactoring; continuous refactoring keeps the design from degrading over time, ensuring that the code is easy to understand, maintain, and change.

\section{Continuous Integration}
Continuous integration reduces the defects in a software system by catching errors early and often and enabling a stop-and-fix process. It leverages both automated acceptance tests and automated developer tests to give frequent feedback to the team and prompts removing these defects promptly.

\section{Collective Code Ownership}
Collective code ownership means that members of a development team have the right and responsibility to modify any part of the code. They get more exposure to the entire code base and are able to remove defects wherever they are found and incrementally modify the design of the system accordingly.

\section{Evolutionary Design}
Evolutionary design is the simple design practice (below) done continuously. Teams start off with a simple design and change that design only when a new requirement cannot be met by the existing design.

\section{Simple Design}
If a decision between coding a design for today’s requirements and a general design to accommodate for tomorrow’s requirements needs to be made, the former is a simple design. Simple design meets the requirements for the current iteration and no more.

\section{Iteration}
An iteration is a time-box where the team builds what is on the backlog and is a potential release and therefore enables building less and forces regularly removing defects to reach the agreed upon done state.

\section{Release Often}
Releasing your software to your end customers as often as you can without inconveniencing them forces you to constantly have your software in releasable quality and allows you to build in smaller increments and get feedback before too much of an investment is made.

\section{Stand Up Meeting}
Stand up meetings are daily meetings for the team to synch-up and share progress and impediments daily. This helps keep the entire team aware of what is being done and where in the system.
\chapter{Understanding how VoIP, Cloud and distributed systems are related}
\label{ch:voipcloud}

\section{Distributed Platform}
The requirements to build a scalable, distributed platform aren’t a mystery to anyone who’s built a distributed system. Even Google has a tutorial on how to do so (http://code.google.com/edu/parallel/dsd-tutorial.html). But most people’s experience in distributed systems revolve around web or database platforms. Distributed VoIP on commodity platforms is still a reasonably new phenomenon (we’ll get into why in a separate post) and it has different requirements.

Generally Speaking, a Scalable, Distributed platform generally consists of these components:

* Messaging: A way for programs across different servers to talk to each other and/or know what other servers are doing

* Redundancy: The ability to have copies of everything (data, software, etc.) all working together at the same time, with copies coming online/offline at any given time

* Distribution of Data: The ability to break information into pieces and spread it across multiple computers, allowing for adding/removing computers as demand requires

* Unlimited Concurrency as a Concept: The idea that there should be no limit to how much is happening on the platform overall 

\section{Specifications of a VoIP platform}
The above are common characteristics. But VoIP is unique in how it works and has additional requirements. Our needs also included:

* Directed Events: The ability for message queues across boxes to be spun up and down quickly and to act as a “tunnel” between different explicit services without disrupting other nodes

* Schema Flexibility: The ability to frequently upgrade data and variable structures within the entire system without bringing down clusters (inherently having different versions of schema running while the system, as a whole, remains operational)

* Strong Supervision: The ability to detect failures *very* quickly and re-spawn nodes and processes just as fast. In web servers, delays and failures of 50ms or more are acceptable - in voice applications, they’re ultimately not

* Speed for Adding Features: Telecom is growing extremely rapidly. The ability to expose new features quickly, in a reliable, scalable, distributed way is paramount to a successful platform

* Fast Server Provisioning: The ability to handle spikes “in the cloud” by procuring and provisioning additional resources (from servers to circuits to DIDs) in an instant

* The ability to move in-progress calls around to servers that have better network connectivity or lower latency

* The ability to avoid ALL downtime (as a goal) - not even upgrades to software should result in downtime.


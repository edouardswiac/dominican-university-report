% A template for master's thesis, W.H. 2006
% If your superviser wants an another template, use it!
% This file contains only the cover page and all chapters are in 
% different files
\documentclass[12pt]{report}

%useful special symbols:
\usepackage{amssymb}
\usepackage{latexsym}

%a useful package if you write url addresses:
\usepackage{url}

%a package for figures:
\usepackage[dvips]{color}
\usepackage{epsfig}

%Default for bibliography style. The alpha style generates references with 
%first letters and year. If your supervisor asks another style, change this. 
\bibliographystyle{alpha}


%Create your own environments
\newtheorem{definition}{Definition}
\newtheorem{example}{Example}

%if you want to use multicolumn tables, uncomment the following:
%\usepackage{multicol}
%if you want to use sideway tables or figures, uncomment the following:
%\usepackage{rotating}

%The following packages are needed e.g. for algorithm environment
\usepackage{float}
\usepackage{xspace}
%Uncomment to include algorithm environment (you need file algorithmwh.sty):
%\usepackage{algorithmwh}

\textwidth=14.945cm
\oddsidemargin=0.5cm
\evensidemargin=0.5cm

%By default, each paragraph begins by an indent (space). If you prefer
%no indent but an empty line between paragraphs, uncomment the following: 
%\setlength{\parindent}{0pt}
%\setlength{\parskip}{1.5ex plus 0.5ex minus 0.2ex}
 
%%%%%  

\begin{document}

%the title page begins
\begin{titlepage}

\vspace*{5cm}

\noindent
\begin{flushleft}
\setlength{\baselineskip}{2\baselineskip}
{{\Huge \bf Scalable VoIP systems in a distributed environment}}

\end{flushleft}

\newcommand{\TitleStyle}[1]{{\Large \textbf{#1}}}
\newcommand{\TitleSpace}{\vspace{2pt}}
\vspace{2cm}
\noindent
{Edouard Swiac}

\vspace{1cm}
\vspace{\fill}
\vspace{2cm}
\begin{tabbing}
mmmmmmmmmmmmmmmmmmmmmmm\= \kill
\>Master's thesis \today\\
\>Department of Computer Science\\
\>SUPINFO International University\\
\> \\
%%\> \\
\end{tabbing}
\end{titlepage}
%the title page ends

%%%%%
\thispagestyle{empty}
%Abstract
\newlength{\origpar}
\setlength{\origpar}{\parindent}
\setlength{\parindent}{0pt}
\begin{abstract}
    Most \emph{Voice over IP} service providers create home-grown tools to manage their systems. As their needs, and their customer's demands, grow, their toolset falls behind and becomes a pain point for their company. A system built to grow with millions of users in mind must be conceived to be scalable from start, and that scalability can be achieved by relying on a distributed architecture. Complexity is abstracted in the Cloud, therefore hidden from the end-user. 
\end{abstract}
\setlength{\parindent}{\origpar}

%if you want to include acknowledgements, uncomment the following:
%\thispagestyle{empty}
%input{acknowledgement}
%Now the file is called acknowledgement.tex
\thispagestyle{empty}
\tableofcontents
\clearpage

\renewcommand {\baselinestretch} {1.0}
\pagebreak 
\pagenumbering{arabic}
\chapter{Introduction}
\label{ch:intro}
\input{chapter1}
\input{chapter2}
%include all chapters here
\input{conclusion}
\newpage
%include appendexes here
%Appendixes do not have section numbers, but they are listed in the table 
%of contents:
\section*{Appendix A: heading}
\label{lablename for referring}
\addcontentsline{toc}{chapter}{Appendix A: heading}

Appendix text here.

%\input{appendixB.tex}

%The following adds references to the table of contents
\addcontentsline{toc}{chapter}{References}

%Give the name of your bibtex database, here it is dbase.bib
\bibliography{dbase}
%If you do not want to use bibtex, comment the previous and uncomment 
%the following, where references.tex contains the references:
%input{references}
\end{document}



